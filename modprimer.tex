\documentclass[letterpaper]{article}

\usepackage{amsthm}
\usepackage{amsmath}
\usepackage{chngcntr}

\theoremstyle{definition}
\newtheorem{theorem}{Theorem}
\newtheorem{definition}{Definition}

\newtheoremstyle{case}{}{}{}{}{}{:}{ }{}
\theoremstyle{case}
\newtheorem{case}{Case}

\counterwithin*{case}{theorem}

\title{A Primer on Modular Arithmetic}
\author{David Sanders}

\begin{document}

\maketitle

First, we define a few simple and familiar classes of numbers.

\begin{definition}
  An \textbf{integer} is a whole number and may be negative, positive, or zero.
\end{definition}

In the above definition, we distinguish zero from the negative and positive
integers.  Thus, zero is considered neither negative nor positive.

\begin{definition}
  The \textbf{positive integers} are those integers that are not negative and not zero.
\end{definition}

\begin{definition}
  The \textbf{non-negative integers} are those integers that are positive or zero.
\end{definition}

The above two classes of integers are distinguished by their inclusion (or not)
of the number zero.

\begin{theorem}[The Division Algorithm]
  \label{division}
  Let $a$ be an integer and $d$ be a positive integer.  Then there are unique
  integers $q$ and $r$, where $0 \le r < d$, such that
  \[ a = dq + r. \]
\end{theorem}

\begin{proof}
  Let $a$ be an integer and $d$ be a positive integer.  Then $a$ is either
  divisible or not divisible by $d$.
  \begin{case}
    If $a$ is divisible by $d$, then $a = dq$ where $q$ is a unique integer.
    In other words, $a$ is equal to a unique multiple of $d$.  Then $a = dq +
    r$ where $r = 0$.
  \end{case}
  \begin{case}
    If $a$ is not divisible by $d$, then $dq < a < d(q + 1)$ where $q$ is a
    unique integer.  In other words, $a$ is between a unique multiple of $d$
    and the next greatest multiple of $d$.  Then,
    \begin{align}
      dq &< a < dq + d \nonumber \\
      0 &< a - dq < d \label{remainder1} \\
      \implies 0 &< r < d. \label{remainder2}
    \end{align}
    From inequality \ref{remainder1}, we see that $a$ differs from $dq$ by a
    positive amount less than $d$.  In inequality \ref{remainder2}, we call
    this amount $r$.  Therefore,
    \begin{align*}
      a - dq &= r \\
      a &= dq + r
    \end{align*}
    where $0 < r < d$.
  \end{case}
  In both cases, $a = dq + r$ where $q$ is unique for $a$ and $0 \le r < d$.
\end{proof}

Why is the Division Algorithm important?  It's important because it shows how
the integers can be split up into a series of ``modules" according to a divisor
$d$.  An integer $a$ can then be mapped to a unique position in a unique module
by a quotient $q$ and a remainder $r$.  It's possible to think of $q$ as being
the ``module number" for $a$ and to think of $r$ as being an offset into that
module.

\end{document}
